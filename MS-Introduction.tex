Germline mutations are estimated to occur in humans with an average probability of $1.28\times10^{-8}$ per site per generation, with $\sim$93\% of these being point mutations \citep{Roach_2010, Jnsson_Parental_2017}. Germline point mutations result in the creation of single nucleotide variants (SNVs) in a population.  Evidence of genomic heterogeneity in mutation has been predominantly derived from between or within species analysis of genetic variation. For instance, mutation heterogeneity is implicitly supported by genomic heterogeneity in substitution rates \citep{Hodgkinson2009,ying2010evidence} and in the relative abundance of nucleotides \citep{cuny1981major}.  More recently, explicit \textit{de novo} mutation studies \citep[e.g.][]{michaelson2012whole, francioli2015genome, smith2018large} have been reported, and these too support a heterogeneity in mutation processes.  The mechanistic origins of this mutation heterogeneity remain unclear.  Likely candidates include a direct mutagenic influence of meiotic recombination and the effect of sequence neighborhood.  Analyses of these potential contributors have predominantly drawn on SNV analyses and have led to inconsistent conclusions. Here we focus on development and application of a consistent analytical framework to quantify the relative importance of these different factors.

It has been established that the rate of mutation is non-uniform along the genome of humans and other species. The phenomenon of mutation heterogeneity was first observed in the bacteriophage T4 prior to the availability of DNA sequencing \citep{benzer1961topography}. Subsequent DNA sequence analyses of homologous genes revealed that G and C nucleotides were far more mutable than A and T nucleotides \citep{Coulondre_1978, gojobori1982patterns} and that mutation rates at these sites are influenced by neighbouring bases \citep{bulmer1986neighboring}. We now have evidence that such non-uniformity can occur at scales ranging from individual nucleotides to multi-megabase sized regions \citep{Hodgkinson_2011}. For instance, the heterogeneity of DNA composition suggests the existence of mutation rate heterogeneity at megabase scales and this has been supported by \textit{de novo} mutation studies \citep{smith2018large}.

Previous studies have identified a number of key determinants of mutation rate, prominent among which are recombination rate and sequence neighbours. However, these studies have differed in certain of their conclusions. For example, relatively recent studies of \textit{de novo} mutations have provided strong evidence of a direct causative effect of recombination on mutation \citep{Arbeithuber_Crossovers_2015}. That nucleotide diversity is higher in regions of high recombination has been known for some time \citep{Lercher_2002, Duret_2008}. Whether this reflects a direct effect of recombination on mutation or an influence of selective sweeps in reducing diversity in regions of lower recombination is disputed \citep[e.g.][]{2019.Jensen.000}. Previous population analyses used linear regression models \citep{Lercher_2002, Duret_2008, mugal2011substitution} to measure an association between mutation and recombination rates. Estimates from these approaches are potentially problematic as the methods used do not control for spatial auto-correlation of recombination and mutation rates across the genome.

The hypermutability of CpG dinucleotides (and the preponderance of genetic variation within this context) exemplifies the important influence of sequence context on the rate of mutation. In mammals and some other species, the transition mutation C\textrightarrow T, where the C is part of a CpG dinucleotide, is several times more common than mutations at other sites \citep{Ehrlich_1981}. The biochemical cause is known to be the spontaneous deamination of the highly unstable 5-methylcytosine \citep{Coulondre_1978}. In mammals, methylation of cytosines is highly context dependent, occurring almost exclusively at CpG dinucleotides \citep{ramsahoye2000non}.

It has been demonstrated that all point mutations are affected to a greater or lesser effect by sequence context \citep{Zhu_2016}. Using a log-linear model, \citet{Zhu_2016} dissected the influence of nucleotide distance and the joint versus independent influence of multiple nucleotides. These authors argued that the dominant neighbourhood influences lay within $\pm2$ for transition mutations, $\pm3$ for transversions. \citet{carlson2018extremely} also found widespread influence of context on mutation types while restricting their analysis to extremely rare variants. \citet{Zhu_2016} and \citet{carlson2018extremely} did not, however, directly address mutation rate or variance in the sense described above.  An analysis using the $R^2$ metric of a linear model to measure the contribution of different contexts to variance \citep{Aggarwala2016} argued that nucleotides up to 3 sites distal can have a major influence on mutation rates. The linear regression model used by Aggarwala and Voight does not yield the maximum likelihood estimates of model parameters for this data, due to the binomial nature of the sampled data and the condition of heteroscedasticity consequently not being satisfied \citep[][p. 120]{agresti}. Previous approaches also did not address issues of bias arising from neighborhood size. Bias will tend to inflate estimates of variance as a given data set of mutation counts is further subdivided into ``buckets'', the number of which increases with neighborhood size $k$ at the rate $4^k$.

One approach to quantifying the relative contributions of different factors on mutation is to measure the proportion of variance in mutation rate explained by them. Conversely, this measurement also indicates how much variance remains unexplained. Inherent in discussion of the variability of mutation rate is the assumption that each site in the genome has a specific mutation rate. Hence, we define the ``total'' variance in mutation rate as the conventional statistical variance of these quantities. This variance has been estimated by comparing variable positions in orthologous alignments of closely related species such as humans and chimpanzees \citep{Hodgkinson2009}. The probability of an SNV at a site is assumed to be some multiple $r$ of the site mutation rate, with $r$ fixed in each population. (The underlying mutation rates are assumed to be the same in humans and chimpanzees.) The variance in mutation rate can then be calculated from the number of SNVs that are observed at orthologous sites in both sequences. The conclusion from this approach was that there was substantial variance in the human mutation rate ($\sim$64\% of total variance) that was not explained by the interaction of a base with its immediately adjacent nucleotides \citep{Hodgkinson2009}. These authors minimized the potential role of larger sequence contexts, a conclusion that was later challenged by the results of other studies \citep{Aggarwala2016, Zhu_2016}.

Here we report work quantifying the contribution to the probability of human genome polymorphism that can be attributed to recombination and to sequence context at different scales. We use a Bayesian approach to quantify the uncertainty in our estimates of the variance and to overcome issues of bias which occur if a conventional estimator were used instead. Our results produce estimates of recombination induced mutation that are consistent with those from \textit{de novo} mutation studies. We further establish that when considered across all point mutations, the influence of sequence neighbourhood is dominated by 5-mer effects reflecting the markedly greater relative abundance of transition mutations. Finally, we emphasize the complexity in comparing the contributions to mutation of a state (sequence context) versus the contribution to mutation of an event (recombination).  Overall, we establish that a substantial proportion of mutation heterogeneity remains unaccounted for.